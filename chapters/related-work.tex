It is no surprise that there are already a number of portable and light sensors designed to collect sensor data about the environment around us. In this section, related work is review and directly compared against the proposed project.

\section{Data Acquisition Apparatus}

There are several proprietary solutions available on the market for gathering sensory information for \acs*{SLAM}, but they tend to be expensive \cite{libackpack_C50}, \cite{libackpack_DGC50}. Additionally, there are some articles and research papers that have been conducted with the goal of building a system similar to the one proposed here. These works will be the ones to be focused on since the methodology, and results are readily available.

In terms of application, Alexander Proudman's system is closest to this project's \cite{proudman_online_2021}. Based on an Ouster OS0-128 \acs{LiDAR} and a RealSense D435i, both with an integrated \acs{IMU}, the system performs online \acs{SLAM} and estimates \acl{DBH} (\acs*{DBH}) using the data collected. While their application has the benefit of having a built-in display that allows real-time visualization of data, one of its major drawbacks is the way they build their apparatus, opting for a metal stick rather than a backpack type of design. As the authors acknowledge, user fatigue may lead to excessive variations in stick position, resulting in unintelligible and uncontrolled movements, damaging the performance of the apparatus.This would not be an issue if a more ergonomic structure was used. The errors due to user fatigue can be a slightly mitigated by performing \acs{SLAM} in several sessions instead of one, allowing the user to rest between shorter sessions \textcolor{red}{INSERT REFERENCE TO THE SECOND PAPER}. After recording multiple sessions, \acs*{GPS} information is used to assemble the multiple sessions's map into a single map.

Recently, Kui Xao developed a dual \acs*{LiDAR}s system, an \acs*{IMU}, and \acs*{GPS} for performing \acs*{SLAM} in multi-scene applications \cite{xiao_high-precision_2022}. To increase the vertical \acl*{FOV} (\acs*{FOV}), one of the \acs*{LiDAR}s was placed in the \textit{XY} plane while the other was positioned at -77.94° from the \textit{XY} plane. A timestamp synchronization algorithm is used by the authors to merge the data of the two \acs*{LiDAR}s. Secondly, the \acs*{LiDAR} data is tightly coupled with the IMU data in order to reduce noise caused by the inaccuracy of \acs*{LiDAR}-based odometry. In outdoor tests, the \acs*{IMU} calibration is improved by loosely coupling \acs*{GPS} to the \acs*{IMU} during outdoor tests.