\section{What is \acs{ROS}?}

An effective robotics project cannot be achieved by just having sensors and physical components; rather, one must have a clever communication system between sensors and processes. Although such complex systems can be built from scratch, it is not worthwhile when there is software like \acs*{ROS} available.

Although the name \acl*{ROS} suggests that ROS is an operating system, this is not the case. \textcolor{red}{Give examples of an OS and shortcomes of ROS in being considered an OS}.  In a way, \acs*{ROS} is both middleware and a framework. The system provides a communication channel where messages can be easily subscribed, published and distributed, allowing quick integration between systems and components. Moreover, it provides features such as debugging, visualization, testing, logging, and configuration right out of the box. Additionally, ROS includes a number of useful packages for essential areas relevant to robotics such as movement, perception, and manipulation. The ROS community is constantly evolving with the most recent developments in robotics, so libraries are always being added to ROS. In robotics, it is considered the standard platform for developing complex projects.

\section{What is \acl*{SLAM} (\acs*{SLAM})?}

Since the early days humans have the interest to map their environment. blah, blah, blah...

\subsection{Classic Algorithms}

\subsection{State of the Art Algorithms}