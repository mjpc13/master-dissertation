Due to their destructive nature, wildfires are often viewed as something to be avoided at all costs, however wildfires play a crucial role in a variety of ecosystems. In addition to rejuvenating the forest, wildfires can increase soil fertility and remove dead organic material, which decreases the likelihood of a more intense and destructive wildfire in the future \cite{bond_fires_2017}. In most fires, the problem occurs when they spiral out of control, causing damage to wildlife and populations. Global warming and an increase in droughts have led to more frequent intensive wildfires in countries that were not historically prone to them. Throughout the Mediterranean region, countries have become accustomed to this natural cycle and have been able to reduce burned areas since 1980 by improving fire control, Portugal remains the exception \cite{turco_decreasing_2016}\cite{european_commission_joint_research_centre_forest_2021}.

In short, wildfires are important to a healthy ecosystem but they need to be controlled to not damage populations and wildlife. The first step to increasing fire control is to clean the forest of dead organic matter, a task that can be performed by robots. As these robots are heavy, scientists and researchers find it difficult to obtain data directly from them, requiring a light system to record the necessary data to develop and test algorithms.

\section{Context}

\section{Objectives}
In this work, we aim to design a rigid multisensory apparatus  with an onboard computer for the mapping of forests that combines multiple sensing technologies, including 3D LIDAR, depth cameras, infrared spectroscopy and inertial measurement units. With this apparatus, we will be able to collect datasets from forest environments, thus supporting forest operations, such as metric-semantic 3D mapping, combustible material identification, and forest cleaning. Aside from designing the apparatus, it is intended to integrate, test, and then compare state-of-the-art 3D mapping techniques based on the ROS middleware.

\section{Requirements}